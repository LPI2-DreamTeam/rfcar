\section{Tests}
\label{sec:org3e2776f}
\subsection{Verifications tests}
\label{sec:orge9c79e2}
The verifications tests are tests performed internally by the design team to
check the compliance of the foreseen specifications. These tests are done after
the prototype alpha is concluded.

\subsubsection{Maximum velocity}
\label{sec:org532616f}
The maximum velocity must be tested to ensure integrity of the
system. Furthermore, the degree of linearity of the velocity is respect to user
input must also be tested.

The procedure to test the maximum velocity is analogous to the velocity's
linearity trial, the only difference being the reference velocity’s regulation potentiometer position which should, for the maximum velocity’s case be at maximum.

The velocity of the conveyor belt can be measured internally to the system ---
measuring the induced voltage at the generator and converting to its physical
representation --- or externally using an tachometer for instantaneous value or
measuring the travel time and dividing by the conveyor length to obtain the
average velocity.

\uline{Internally}: The motor drives the conveyor belt at one end, also driving the
generator at the other end, due to their coupling. Thus, disregarding slip and
friction losses, the angular velocity of the motor and the generator are
identical. As \(\omega_m\) = \(\omega_g\) and \(v = \omega \cdot r\), the linear velocity of the conveyor can determined based on the angular velocity of the generator, which induces a proportional voltage at the terminals of the generator. Thus, by measuring this voltage, the linear velocity of the conveyor can be determined.

Using an oscilloscope to display the measured voltage at the generator and the
reference voltage, one will have the electrical representation of both the
desired velocity (physical representation of the reference voltage) and the
current velocity (physical representation of the voltage at the generator). 
One should see them as very similar, within a previously agreed upon range of
difference, accordingly to the type of controller used. This can be be used for
transient analysis of the conveyor's behaviour.

\uline{Externally} (instantaneous value): This first method will make use of the
physical relation between the linear velocity, v, the angular velocity,
\(\omega\), and the radius of the axis, r: \(v = \omega \cdot r\); and the previously
stated assumption that the conveyor’s linear velocity and the generator’s
angular velocity are directly proportional.

Using a tachometer to measure the motor's angular
velocity \(\omega\), the linear velocity of the conveyor belt can be determined,
through \(v= \omega \cdot r\), where \(r\) is the radius of the axis. 

Then through a comparison of the measured velocity and the physical representation of the reference voltage the outcome of the trial will be clear: if the velocities are similar within a range of difference that was agreed upon, it should be considered a success.

\uline{Externally} (average value): measuring the travel time (see Section
\ref{sec:org20789b4}) of a part in the conveyor and
dividing by the conveyor length, the average velocity of the conveyor can be
determined. This can only be used for steady state analysis of the conveyor's
behaviour. Thus, if the average velocity is within the boundaries of the desired
velocity and the respective margin of error, the trial is considered a success.

For maximum assurance one should at least measure the velocity through the
internal method and one of the external followed by a comparison. This
comparison takes into account that for this procedure we agreed upon a 5\% margin
of error when comparing the measured and the reference velocities. It also
should consider the overshoot that will occur when the load is first placed,
which was agreed to be 10\%, therefore at any given time the conveyor’s velocity
should never surpass \texttt{0.2 m/s} (the agreed upon maximum velocity) \(\pm\) 10\%.

\subsubsection{Travel Time}
\label{sec:org20789b4}
The time it takes a certain load with a constant mass to travel the full length
of the conveyor can be measured by a simple series of measurements using a
chronometer and the calculation of the average.

It should be noted that during the external measurement of velocity, the travel time was measured, there is a direct correlation between the two as such at assurance’s behest one should make sure that the results obtained during the velocity trial are in accordance with the values obtained in this procedure.

\subsubsection{Settling time}
\label{sec:orgf4c025f}
The time that it takes the system to react to the presence of a load with a
constant mass and achieve a steady state.

The small scale of the conveyor and, consequently, its settling time (0.6
seconds minimum for the maximum velocity) dictate that the easiest method to
measure the conveyor velocity is by measuring the induced voltage in the
conveyor, capturing this data using the oscilloscope, this taking into account,
once again, the relation between the induced voltage at the generator and the
conveyor’s current velocity, as such by observing the induced voltage’s behavior
one can draw a conclusion regarding the settling time.

With the induced voltage at the generator captured by an oscilloscope,
specifically using the “single mode” present in these measuring instruments one
can observe the change that will occur in the generator’s voltage, more
specifically the moment a load is placed upon the running conveyor a change will
occur, the time it takes from this moment until the voltage returns to previous
value will be the settling time.

\subsubsection{Overshoot}
\label{sec:orgb1f5c2a}
An overshoot occurs when the output in a control system exceeds its final,
steady state value generally caused by a sudden change in the system, in this
case specifically, the placement of a load upon the conveyor will cause an
overshoot in the latter’s velocity which must be controlled lest it cause
problems.

An overshoot will occur during the settling time, as such, using the same
considerations taken in its measurement, it can be measured by observing the
induced voltage at the generator (an overshoot in the conveyor’s velocity will
correspond to a peak in the generator’s voltage).

Using an oscilloscope to display the induced voltage at the generator and making
use of the “single mode” present in these measuring instruments one can observe
the change that will occur in the generator’s induced voltage, the peak voltage
that will be seen when the load is placed upon the running conveyor is the
electrical representation of the overshoot of velocity, then either by
converting it to its physical representation or comparing it to the reference
voltage one can arrive at a conclusion. It was agreed that the overshoot
velocity should be \(V_{ss} \pm 10\%\), where \(V_{ss}\) is the stedy state velocity.

\subsection{Validation tests}
\label{sec:orgff1a37d}
The validation tests should be performed by the client using the product’s
manual, so it is expected that a user without prior experience with the product
should be able to use it correctly and safely. 
For this purpose, a laboratory guide will be produced to work similarly to the
product’s manual, so the user can use the product also as a didactic kit. However, in this stage of the project, the laboratory guide hasn’t been prepared yet, but it’s already possible to have an idea about the type of tests and steps that will be necessary, such as:
\begin{itemize}
\item Motor tests;
\item Generator tests;
\item System parameter determination by measurements;
\item Tests using another controller;
\item Tests varying the load;
\item Tests changing slope levels;
\end{itemize}
%%% Local Variables:
%%% mode: latex
%%% TeX-master: "../Phase1"
%%% End:
