\section{Foreseen product specifications}%
\label{sec:org31f7574}
The foreseen product specifications are listed as topics below (sketch).

\subsection{Autonomy}%
\label{sec:orge975868}
Check Battery and power consumption: How long does it take to completely discharge the battery taking into account the energy consumption.
\subsection{Velocity}%
\label{sec:org08718bc}
Maximum Velocity allowed for the car to achieve in ideal conditions.
\subsection{Safety}%
\label{sec:org83942c3}
\begin{itemize}
\item Car: If the user issues a command that would cause damage to the system, the
system should take corrective measures to prevent it. The same holds true if
the communication between user and system is lost.
\begin{itemize}
\item \textbf{System uses odometric navigation}
\end{itemize}
\item Human: preserve human safety
\end{itemize}
\subsection{Image acquisition}%
\label{sec:orgb6a5f66}
\subsubsection{Frame rate}%
\label{sec:org5adf4ee}
Frequency at which independent still images appear on the screen.
\subsubsection{Range}%
\label{sec:orgecb044c}
How far can the camera capture images and record them.
\subsubsection{Resolution}%
\label{sec:orgba87554}
The amount of detail that the camera can capture. It is measured in pixels. The quality of the aquired image is proportional to the number os pixels.
\subsubsection{Color scale (Black and white or color)}%
\label{sec:org468ee04}
??
\subsubsection{Always present or enabled on user command}%
\label{sec:orgd585352}
??
\subsection{Usability}%
\label{sec:org61632e0}
\begin{itemize}
\item User-friendly interface
\item User interface responsiveness
\end{itemize}
\subsection{Load}%
\label{sec:orgca6a690}
Maximum load the car can safely carry.
\subsection{Overall System latency/responsivess}%
\label{sec:org7fd1829}
The overall system latency is the sum of all systems' latencies, which must be
under a maximum tolerated value for the user.
\subsection{Communication}%
\label{sec:org4241610}
\subsubsection{Reliability}%
\label{sec:orgdcb920d}
Packet must be delivered (reliable, e.g. TCP) or not (e.g. UDP)
\subsubsection{Range}%
\label{sec:org447a205}
Maximum distance allowed between user and system for communication purposes
\subsubsection{Transmission rate / throughput}%
\label{sec:org10e75a5}
\subsubsection{Redundancy}%
\label{sec:orgc5933fc}
\subsection{Sensibility}%
\label{sec:org622e63a}
Sensibility to Smartphone motion
\subsubsection{Msg Smartphone->Raspberry}%
\label{sec:org6b5cb97}
x10 y20 v10
t5 v5

\noindent\rule{\textwidth}{0.5pt}
\subsection{Closed loop error (Control team)}%
\label{sec:org436f732}
Error associated
\subsubsection{PI}%
\label{sec:org9859444}%
\subsubsection{PID}%
\label{sec:org352c4d4}
\subsubsection{PD}%
\label{sec:org0d324c4}

\subsection{Summary}%
\label{sec:org1f95256}
Table~\ref{tab:specs-init} lista the foreseen product specifications.

% Please add the following required packages to your document preamble:
\begin{table}[!hbt]
\centering
\caption{Specifications}%
\label{tab:specs-init}
\resizebox{\textwidth}{!}{%
\begin{tabular}{lll}
\hline
 & Values & Explanation \\ \hline
Max Velocity & 0.2 m/s & Maximum velocity of the conveyor belt in steady state \\ \hline
Dimensions & 60x30x30 cm & Dimensions of the conveyor belt in cm {[}l w h{]} \\ \hline
Time Min & 3 s & \begin{tabular}[c]{@{}l@{}}Time taken to transport a load the full extent \\ of the conveyor belt at maximum velocity\end{tabular} \\ \hline
Max Load & 1 Kg & \begin{tabular}[c]{@{}l@{}}Load the belt can hold without causing any \\ harm to the product\end{tabular} \\ \hline
Max slope & 15$^\circ$ & \begin{tabular}[c]{@{}l@{}}Maximum slope in which the conveyor belt can \\ operate at nominal conditions\end{tabular} \\ \hline
Slope levels & {[}0,5,10,15{]}$^\circ$ & Different levels of slope manually handled \\ \hline
Settling time & 0.2 $\cdot$ T & \begin{tabular}[c]{@{}l@{}}This means that it takes up to 20\% of the full \\ travel time to reach steady velocity\end{tabular} \\ \hline
Overshoot & 110\% Vss & \begin{tabular}[c]{@{}l@{}}Maximum velocity the conveyor belt reaches \\ before settling time\end{tabular} \\ \hline
Margin of error & 95\%-105\% of Vss & Admissible error in steady velocity \\ \hline
Power supply & 12V batteries, 6W & The main power supply will be 12V batteries \\ \hline
\end{tabular}%
}
\end{table}

%%% Local Variables:
%%% mode: latex
%%% TeX-master: "../Phase1"
%%% End:
