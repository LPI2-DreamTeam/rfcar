% Created 2020-06-30 Tue 07:07
% Intended LaTeX compiler: pdflatex
\documentclass[11pt]{article}
\usepackage[utf8]{inputenc}
\usepackage[T1]{fontenc}
\usepackage{graphicx}
\usepackage{grffile}
\usepackage{longtable}
\usepackage{wrapfig}
\usepackage{rotating}
\usepackage[normalem]{ulem}
\usepackage{amsmath}
\usepackage{textcomp}
\usepackage{amssymb}
\usepackage{capt-of}
\usepackage{hyperref}
\usepackage{color}
\usepackage{listings}
\usepackage[table,xcdraw]{xcolor}
\usepackage{listings}
% src: http://texdoc.net/texmf-dist/doc/latex/listings/listings.pdf
% c/c++
\lstdefinestyle{customc}{
  belowcaptionskip=1\baselineskip,
  breaklines=true,
  %frame=L,%lines, whole
  xleftmargin=\parindent,
  language=C,
  showstringspaces=false,
  basicstyle=\scriptsize\ttfamily,
  keywordstyle=\bfseries\color{green!40!black},
  commentstyle=\itshape\color{purple!40!black},
  identifierstyle=\color{blue},
  stringstyle=\color{orange},
  numberstyle={\tiny},
  numbers=left,
  numberblanklines=false,
  stepnumber=5,
  backgroundcolor=\color{yellow!10}, 
  frame=tlb
}
\lstset{escapechar=@,style=customc}
% ASM
\lstdefinestyle{customasm}{
  belowcaptionskip=1\baselineskip,
  frame=L,
  xleftmargin=\parindent,
  language=[x86masm]Assembler,
  basicstyle=\footnotesize\ttfamily,
  commentstyle=\itshape\color{purple!40!black},
}
% MATLAB: src: https://tex.stackexchange.com/a/75124
\definecolor{greenmb}{RGB}{28,172,0} % color values Red, Green, Blue
\definecolor{lilasmb}{RGB}{170,55,241}
\lstdefinestyle{custom-matlab}{
  % \lstset{language=Matlab,%
  language=Matlab,
    %basicstyle=\color{red},
    breaklines=true,%
    morekeywords={matlab2tikz},
    keywordstyle=\color{blue},%
    morekeywords=[2]{1}, keywordstyle=[2]{\color{black}},
    identifierstyle=\color{black},%
    stringstyle=\color{lilasmb},
    commentstyle=\color{greenmb},%
    showstringspaces=false,%without this there will be a symbol in the places where there is a space
    numbers=left,%
    numberstyle={\tiny \color{black}},% size of the numbers
    numbersep=9pt, % this defines how far the numbers are from the text
    emph=[1]{for,end,break},emphstyle=[1]\color{red}, %some words to emphasise
    %emph=[2]{word1,word2}, emphstyle=[2]{style},    
}
% Java: src: 
\lstdefinestyle{custom-java}{
	language=Java,
	tabsize = 4, %% set tab space width
	showstringspaces = false, %% prevent space marking in strings, string is defined as the text that is generally printed directly to the console
	numbers = left, %% display line numbers on the left
	commentstyle = \color{green}, %% set comment color
	keywordstyle = \color{blue}, %% set keyword color
	stringstyle = \color{red}, %% set string color
	rulecolor = \color{black}, %% set frame color to avoid being affected by text color
	basicstyle = \small \ttfamily , %% set listing font and size
	breaklines = true, %% enable line breaking
	numberstyle = \tiny,
}

\author{José Miguel Alves Pires\thanks{a50178@alunos.uminho.pt}}
\date{\textit{<2020-06-30 Tue>}}
\title{cheatsheet}
\hypersetup{
 pdfauthor={José Miguel Alves Pires},
 pdftitle={cheatsheet},
 pdfkeywords={},
 pdfsubject={},
 pdfcreator={Emacs 26.3 (Org mode 9.3.6)}, 
 pdflang={English}}
\begin{document}

\maketitle
\tableofcontents


\section{Modular design\hfill{}\textsc{Important}}
\label{sec:org400fc2f}
Modular design is useful to separate logical units inside the \LaTeX{} environment.
\begin{itemize}
\item \texttt{dissertation.tex}: main document that defines all the formatting and includes
all the relevant \texttt{.tex} documents
\item \texttt{./sty}: contains the stylesheets for the document, such as:
\begin{itemize}
\item \texttt{dissertation-xelatex.sty}: stylesheet for the main document
\item \texttt{listing.sty}: stylesheet for the listings to be formatted and presented
\end{itemize}
\item \texttt{./sec}: contains secondary files, like images, PDFs, acronyms and symbols
\item \texttt{./bib}: contains the bibliography database
\item \texttt{./listing}: contains the listings (code) to be displayed
\item \texttt{./font}: contains the fonts to be used
\item \texttt{./tex}: contains the \texttt{.tex} documents to be inputted, subdivided in:
\begin{itemize}
\item \texttt{Pre\_Chap}: Pre chapters (not used here)
\item \texttt{Chap}: chapters
\item \texttt{Append}: appendices
\end{itemize}
\end{itemize}

Regarding \texttt{.tex} documents, they can be included as follows (the extension
\texttt{.tex} is implicit):
\begin{itemize}
\item \texttt{\textbackslash{}input\{path/to/file\}}: inputs the file as is (\textsubscript{recommended}\_)
\item \texttt{\textbackslash{}include\{path/to/file\}}: inputs the file and adds an extra blank page at the
end.
\end{itemize}

\section{Styling guides\hfill{}\textsc{Important}}
\label{sec:orga642a07}
\begin{enumerate}
\item Don't use \texttt{\textbackslash{}newpage}, \texttt{\textbackslash{}clearpage}, or any other section break command. This
is done natively by the \TeX{} engine; it should be done last and only if REALLY
required
\item To separate paragraphs use one blank line; any additional blank lines MUST BE
AVOIDED anywhere in the document. If you want to separate anything visually,
use a comment \texttt{\%}.
\begin{itemize}
\item Example:
\lstset{language=[LaTeX]TeX,label= ,caption= ,captionpos=b,numbers=none}
\begin{lstlisting}
%
\section{Product concept}%
\label{sec:product-concept}
\section{Product concept: Radio Frequency Camera Assisted Rover (RFCAR)}
\label{sec:orge7b0dc6}
The envisioned product consists of a remote controlled car used to assist
exploration and maintenance domains. For this purpose, the vehicle should contain a
remotely operated camera feeding back video to the user.
Additionally, the
vehicle must contain odometric sensors to assist in driving and prevent
crashes when user is not in control, e.g., when connection is lost.
The vehicle can be used for exploration of unaccessible areas to human operators
like fluid pipelines and other hazardous sites.
I've edited this (Zé).

%%% Local Variables:
%%% mode: latex
%%% TeX-master: "../Phase1"
%%% End:

%
\section{Foreseen specifications}%
\label{sec:fores-spec}
\section{Foreseen product specifications}
\label{sec:org31f7574}
The foreseen product specifications are listed as topics below.

\subsection{Autonomy}
\label{sec:org7364ba5}
The vehicle is operated off-the-grid, thus, a portable power source must be included. The autonomy referes to the time interval between battery fully charged and safely discharged and should be observed for the following scenarios:
\begin{itemize}
\item No load;
\item Vehicle operating at maximum speed;
\item Vehicle operating at minimum speed.
\end{itemize}
\subsection{Velocity}
\label{sec:org08718bc}
The vehicle must be operated within a safe range of velocity, while also not increasing excessively the power consumption. Thus, these velocity boundaries should be tested in the absence of an external load and in the presence of the maximum load.
\subsection{Safety}
\label{sec:org83942c3}
For a remote controlled car, safety concerns not only the car itself and all of the equipment, but also the humans that interact with the car:
\begin{itemize}
\item Car: If the user issues a command that would cause damage to the system, the
system should take corrective measures to prevent it. The same holds true if
the communication between user and system is lost. The system uses odometric navigation.
\item Human: Due to the odometric sensors safely fixed in the car, crashes will not occur, making it much harder for the car to hit a person or for any part of the car to jump and cause harm to the user or anyone around.
\end{itemize}
\subsection{Image acquisition}
\label{sec:orgb6a5f66}
\subsubsection{Frame rate}
\label{sec:org5adf4ee}
Frame rate refers to the frequency at which independent still images appear on the screen. The higher the frame rate, a better image quality is obtained but the processing overhead increases as well, so a compromise must be achieved between the quality of the image and the processing overhead required. The minimum frame rate defined must be such that allows a clear view of the video.
\subsubsection{Range}
\label{sec:orgecb044c}
How far can the camera capture images without loosing resolution and record them. The range must such that allows the user to see the obstacles when the car is heading to them and provide enough time to change the direction.
\subsubsection{Resolution}
\label{sec:orgba87554}
The amount of detail that the camera can capture. It is measured in pixels. The quality of the aquired image is proportional to the number os pixels but a greater resolution requires a greater data transfer and processing overhead, thus, a compromise must be achieved. The minimum resolution must be such that provides the least amount of information required for the user. 
\subsection{Load}
\label{sec:orgca6a690}
The remotely controlled vehicle can be used in applications involving load carrying (besides its own), e.g., packet delivery. For this purpose it is important to determine the maximum load the vehicle can carry safely at the minimum velocity defined. As the load increases, also increases the power consumption, diminishing the autonomy.
\subsection{Overall System latency/responsivess}
\label{sec:org7fd1829}
The overall system latency is the sum of all systems' latencies, which must be under a maximum tolerated value for the user.
\subsection{Communication}
\label{sec:org4241610}
\subsubsection{Reliability}
\label{sec:orgdcb920d}
A communication is reliable if it guarantees measures to deliver the data conveyed in the communication link. As reliability imposes these measures, it also adds overhead to the communication protocol, which must be considered depending on the case. For example, for the devised product, an user command must be acknowledged to be processed, otherwise, the user must be informed; on the other hand, loosing frames from the video feed is not so critical — user can still observe conveniently the field of vision if the frame rate is within acceptable boundaries.
\subsubsection{Range}
\label{sec:org447a205}
The communication protocols have a limited range of operation, and, as such, regarding the environment on which the car is used the range can be changed.
The range refers to the maximum distance allowed between user and system for communication purposes.
\subsection{Sensibility}
\label{sec:org622e63a}
The movement of the car will be determined by the tilt movement of the smartphone. Sensibility refers to the responsiveness of the car on the minimum smartphone tilt movement. The sensibility must be in an range of values in which small unintentional movements will be enough to change the state of the car and it does not take big smartphone tilts for the car to move.
\subsection{Closed loop error}
\label{sec:org436f732}
The velocity, direction and distance to obstacles must be continuosly monitored to ensure proper vehicle operation. The closed loop error must then be checked mainly in three situations as a response to an user command:
\begin{itemize}
\item velocity: the user issued an command with a given mean velocity, which should be compared with the steady-state mean velocity of the vehicle.
\item direction: the user issued an command with a given direction, which should be compared to the vehicle direction.
\item distance to obstacles: the user issued an command with a given direction and velocity which can cause it to crash. The local control must take over control, preventing this to happen, and the final distance to the obstacles must be assessed and compared to the defined one.
\end{itemize}
\subsection{Summary}
\label{sec:org1f95256}
Table \ref{tab:specs-init} lists the foreseen product specifications.

% Please add the following required packages to your document preamble:
% \usepackage[table,xcdraw]{xcolor}
% If you use beamer only pass "xcolor=table" option, i.e. \documentclass[xcolor=table]{beamer}
% Please add the following required packages to your document preamble:
% \usepackage[table,xcdraw]{xcolor}
% If you use beamer only pass "xcolor=table" option, i.e. \documentclass[xcolor=table]{beamer}
\begin{table}[!hbt]
\centering
\caption{Specifications}
\label{tab:specs-init}

\begin{tabular}{
>{\columncolor[HTML]{FFFFFF}}l 
>{\columncolor[HTML]{FFFFFF}}l 
>{\columncolor[HTML]{FFFFFF}}l }
\hline
                  & Values     & Explanation                                                                                                  \\ \hline
Autonomy          & 4 h        & \begin{tabular}[c]{@{}l@{}}Time interval between battery fully \\ charged and safely discharged\end{tabular} \\ \hline
Minimum Velocity  & 0.1 m/s    & \begin{tabular}[c]{@{}l@{}}Minimum velocity at which the car \\ can operate safely\end{tabular}              \\ \hline
Maximum Velocity  & 1 m/s      & \begin{tabular}[c]{@{}l@{}}Maximum velocity at which the car\\ can operate safely\end{tabular}               \\ \hline
Maximum Load      & 0.5 kg     & Maximum load the car can safely carry                                                                        \\ \hline
Frame Rate        & 60 fps     & \begin{tabular}[c]{@{}l@{}}Frequency at which independent still \\ images appear on the screen\end{tabular}  \\ \hline
Camera Range      & 20 m       & \begin{tabular}[c]{@{}l@{}}How far can the camera capture images\\ without loosing resolution\end{tabular}   \\ \hline
Camera resolution & 480p       & Amount of detail that the camera can capture                                                                 \\ \hline
Comunication Range & 50 m & \begin{tabular}[c]{@{}l@{}}Maximum distance between the car and the\\ smarphone without losing connection\end{tabular} \\ \hline
Velocity Error    & 5 \%       & \begin{tabular}[c]{@{}l@{}}Maximum difference between desired \\ and real velocity\end{tabular}              \\ \hline
Direction Error   & 5\%        & \begin{tabular}[c]{@{}l@{}}Maximum difference between desired\\  and real direction\end{tabular}             \\ \hline
Distance Error     & 5 \% & \begin{tabular}[c]{@{}l@{}}Maximum difference between desired\\ and real distance to the obstacle\end{tabular}         \\ \hline
Dimensions        & 20x12x5 cm & Dimensions of the car                                                                                        \\ \hline
Weight            & 0.5 kg     & Weight of the car                                                                                            \\ \hline
\end{tabular}
\end{table}

%%% Local Variables:
%%% mode: latex
%%% TeX-master: "../Phase1"
%%% End:

\end{lstlisting}
\end{itemize}
\item If adding a figure, table, listing, acronym/symbol or citation, or any
sectioning command, please add a label so it can be referenced later. Check
the appropriate section for more info.
\item When adding the reference for the label (see \hyperref[sec:org4434181]{here} for more info), use
appropriate reference designator before referencing using an non-breaking
space in between.
\begin{itemize}
\item Example:
\lstset{language=[LaTeX]TeX,label= ,caption= ,captionpos=b,numbers=none}
\begin{lstlisting}
Fig.~\ref{fig:initial-design}
\end{lstlisting}
\end{itemize}
\end{enumerate}
\section{Sectioning}
\label{sec:orgceb07cf}
Sectioning is used to define the logical structure of the document. The most
relevant commands are:
\begin{enumerate}
\item \texttt{\textbackslash{}chapter}: chapter
\item \texttt{\textbackslash{}section}: section
\item \texttt{\textbackslash{}subsection}: subsection
\item \texttt{\textbackslash{}subsubsection}: subsubsection
\item \texttt{\textbackslash{}paragraph}: paragraph
\item \texttt{\textbackslash{}subparagraph}: subparagraph
\end{enumerate}

Example:
\lstset{language=[LaTeX]TeX,label= ,caption= ,captionpos=b,numbers=none}
\begin{lstlisting}
\section{Product concept}%
\label{sec:product-concept}
\section{Product concept: Radio Frequency Camera Assisted Rover (RFCAR)}
\label{sec:orge7b0dc6}
The envisioned product consists of a remote controlled car used to assist
exploration and maintenance domains. For this purpose, the vehicle should contain a
remotely operated camera feeding back video to the user.
Additionally, the
vehicle must contain odometric sensors to assist in driving and prevent
crashes when user is not in control, e.g., when connection is lost.
The vehicle can be used for exploration of unaccessible areas to human operators
like fluid pipelines and other hazardous sites.
I've edited this (Zé).

%%% Local Variables:
%%% mode: latex
%%% TeX-master: "../Phase1"
%%% End:

\end{lstlisting}
\section{Floats}
\label{sec:org291797e}
\subsection{Figures}
\label{sec:orgb6187de}
\lstset{language=[LaTeX]TeX,label= ,caption= ,captionpos=b,numbers=none}
\begin{lstlisting}
\begin{figure}[!ht]
\centering
\includegraphics[width=1.0\textwidth]{./img/initial_design_diagram.png}
\caption{\label{fig:initial-design}Initial design: Block diagram view}
\end{figure}
\end{lstlisting}
\textbf{Result} (see Fig. \ref{fig:initial-design}):
\begin{figure}[!ht]
\centering
\includegraphics[width=1.0\textwidth]{./img/initial_design_diagram.png}
\caption{\label{fig:initial-design}Initial design: Block diagram view}
\end{figure}
\subsection{Tables}
\label{sec:orgd573a75}
\begin{enumerate}
\item Construction: Tables can be created using an online tool
(\url{https://www.tablesgenerator.com/}) and then copied, as illustrated in
\emph{Definition}
\item Definition: 
\lstset{language=[LaTeX]TeX,label= ,caption= ,captionpos=b,numbers=none}
\begin{lstlisting}
% Please add the following required packages to your document preamble:
% \usepackage[table,xcdraw]{xcolor}
\begin{table}[!hbt]
\centering
\caption{Specifications}%
\label{tab:specs-init}
%
\begin{tabular}{
>{\columncolor[HTML]{FFFFFF}}l 
>{\columncolor[HTML]{FFFFFF}}l 
>{\columncolor[HTML]{FFFFFF}}l }
\hline
		& Values     & Explanation                                                                                                  \\ \hline
Autonomy          & 4 h        & \begin{tabular}[c]{@{}l@{}}Time interval between battery fully \\ charged and safely discharged\end{tabular} \\ \hline
Speed Range  & 0.1 to 1 m/s    & \begin{tabular}[c]{@{}l@{}}Speed at which the car can operate\end{tabular}              \\ \hline
Frame Rate        & 60 fps     & \begin{tabular}[c]{@{}l@{}}Frequency at which independent still \\ images appear on the screen\end{tabular}  \\ \hline
Camera Range      & 20 m       & \begin{tabular}[c]{@{}l@{}}How far can the camera capture images\\ without loosing resolution\end{tabular}   \\ \hline
Camera resolution & 480p       & Amount of detail that the camera can capture                                                                 \\ \hline
Communication Range & 50 m & \begin{tabular}[c]{@{}l@{}}Maximum distance between the car and the\\ smarphone without losing connection\end{tabular} \\ \hline
speed Error    & 5 \%       & \begin{tabular}[c]{@{}l@{}}Maximum difference between desired \\ and real speed\end{tabular}              \\ \hline
Direction Error   & 5\%        & \begin{tabular}[c]{@{}l@{}}Maximum difference between desired\\  and real direction\end{tabular}             \\ \hline
Distance Error     & 5 \% & \begin{tabular}[c]{@{}l@{}}Maximum difference between desired\\ and real distance to the obstacle\end{tabular}         \\ \hline
Dimensions        & 20x12x5 cm & Dimensions of the car                                                                                        \\ \hline
Weight            & 0.5 kg     & Weight of the car                                                                                            \\ \hline
\end{tabular}
\end{table}
\end{lstlisting}
\item \textbf{Result}: Table \ref{tab:specs-init} lists the foreseen product
specifications.
\begin{table}[!hbt]
\centering
\caption{Specifications}%
\label{tab:specs-init}
%
\begin{tabular}{
>{\columncolor[HTML]{FFFFFF}}l 
>{\columncolor[HTML]{FFFFFF}}l 
>{\columncolor[HTML]{FFFFFF}}l }
\hline
		& Values     & Explanation                                                                                                  \\ \hline
Autonomy          & 4 h        & \begin{tabular}[c]{@{}l@{}}Time interval between battery fully \\ charged and safely discharged\end{tabular} \\ \hline
Speed Range  & 0.1 to 1 m/s    & \begin{tabular}[c]{@{}l@{}}Speed at which the car can operate\end{tabular}              \\ \hline
Frame Rate        & 60 fps     & \begin{tabular}[c]{@{}l@{}}Frequency at which independent still \\ images appear on the screen\end{tabular}  \\ \hline
Camera Range      & 20 m       & \begin{tabular}[c]{@{}l@{}}How far can the camera capture images\\ without loosing resolution\end{tabular}   \\ \hline
Camera resolution & 480p       & Amount of detail that the camera can capture                                                                 \\ \hline
Communication Range & 50 m & \begin{tabular}[c]{@{}l@{}}Maximum distance between the car and the\\ smarphone without losing connection\end{tabular} \\ \hline
speed Error    & 5 \%       & \begin{tabular}[c]{@{}l@{}}Maximum difference between desired \\ and real speed\end{tabular}              \\ \hline
Direction Error   & 5\%        & \begin{tabular}[c]{@{}l@{}}Maximum difference between desired\\  and real direction\end{tabular}             \\ \hline
Distance Error     & 5 \% & \begin{tabular}[c]{@{}l@{}}Maximum difference between desired\\ and real distance to the obstacle\end{tabular}         \\ \hline
Dimensions        & 20x12x5 cm & Dimensions of the car                                                                                        \\ \hline
Weight            & 0.5 kg     & Weight of the car                                                                                            \\ \hline
\end{tabular}
\end{table}
\end{enumerate}
\section{Referencing}
\label{sec:org4434181}
To reference the relevant items such as figures, tables, sections (chapter,
section, subsection), etc., one needs:
\begin{enumerate}
\item To label the item, using \texttt{\textbackslash{}label\{<item>:<item-name>\}}:
e.g. \texttt{\textbackslash{}label\{ch:analysis\}}
\item Then, one can reference it using: \texttt{\textbackslash{}ref\{<item>:<item-name>\}}:
e.g. Chapter\textasciitilde{}\ref{ch:analysis}
\end{enumerate}
\section{Bibliography}
\label{sec:orge62b455}
Bibliography management is composed of 2 parts:
\begin{enumerate}
\item A Bibliography database, generally a \texttt{.bib} file, located at
\texttt{./bib/dissert.bib} (see \href{file:///Users/zemiguel/Documents/Univ/MI\_Electro/Sem6/LPI2/PI/github/Deliverables/Final/bib/dissert.bib}{here})
\begin{itemize}
\item Example of Bibliography entry
\lstset{language=[LaTeX]TeX,label= ,caption= ,captionpos=b,numbers=none}
\begin{lstlisting}
@article{harashima1996mechatronics,
  title={Mechatronics-" What Is It, Why, and How?" An Editorial},
  author={Harashima, Fumio and Tomizuka, Masayoshi and Fukuda, Toshio},
  journal={IEEE/ASME Transactions on Mechatronics},
  volume={1},
  number={1},
  pages={1--4},
  year={1996},
  publisher={IEEE}
}
\end{lstlisting}
\item Bibliography entry can be created as:
\begin{itemize}
\item Search the topic at \url{https://scholar.google.pt/}.
\item Select the Export To Bibtex option
\item Copy to the \texttt{.bib} file
\end{itemize}
\end{itemize}
\item A citation, using \texttt{\textbackslash{}cite\{<bib-key>\}}, where \texttt{<bib-key>} is the key defined in
the \texttt{.bib} file. 
\begin{itemize}
\item For example: Mechatronics, was defined by Harashima
et. al=\textasciitilde{}\cite{harashima1996mechatronics}=
\end{itemize}
\end{enumerate}
\section{Enviroments}
\label{sec:org4102c27}
\subsection{Itemize}
\label{sec:org574551c}
\lstset{language=[LaTeX]TeX,label= ,caption= ,captionpos=b,numbers=none}
\begin{lstlisting}
\begin{itemize}
\item \textbf{Item 1}: this is an item
\item \textbf{Item 2}: this is another item
\end{itemize}
\end{lstlisting}
\textbf{Result}:
\begin{itemize}
\item \textbf{Item 1}: this is an item
\item \textbf{Item 2}: this is another item
\end{itemize}

\subsection{Enumerate}
\label{sec:org9d11cf3}
\lstset{language=[LaTeX]TeX,label= ,caption= ,captionpos=b,numbers=none}
\begin{lstlisting}
\begin{enumerate}
\item \textbf{Item 1}: this is an enumerated item
\item \textbf{Item 2}: this is another enumerated item
\end{enumerate}
\end{lstlisting}
\textbf{Result}:
\begin{enumerate}
\item \textbf{Item 1}: this is an enumerated item
\item \textbf{Item 2}: this is another enumerated item
\end{enumerate}
\section{Glossary}
\label{sec:org8845f59}
Glossaries are useful to input \uline{acronyms} and \uline{symbols}.
\begin{itemize}
\item Acronyms: common use words, often abbreviated.
\item Symbols: mathematical/physical symbols that usually require some brief
description and the relevant units.
\end{itemize}

Glossary management is composed of 3 parts:
\begin{enumerate}
\item A Glossary database 
\begin{itemize}
\item Acronyms: \texttt{./sec/acronyms.tex}
\item Symbols: \texttt{./sec/symbols.tex}
\end{itemize}
\item A reference using \texttt{\textbackslash{}gls\{<gls-key>\}}, where \texttt{<gls-key>} is the key defined in
the glossary database file.
\item An external utility that manages the glossary entry items addition and
referencing (no need to worry about this, the makefile will handle it).
\end{enumerate}
\subsection{Acronyms}
\label{sec:orgc5e2914}
\begin{itemize}
\item Definition (\texttt{./sec/acronyms.tex}):
\lstset{language=[LaTeX]TeX,label= ,caption= ,captionpos=b,numbers=none}
\begin{lstlisting}
\newacronym{sls}{SLS}{Selective Laser Sintering}
\newacronym{slm}{SLM}{Selective Laser Melting}
\end{lstlisting}
\item Usage: These are two acronyms used together: \texttt{\textbackslash{}gls\{sls\}/\textbackslash{}gls\{slm\}} technology.
\end{itemize}

\subsection{Symbols}
\label{sec:orgd66ae0b}
\begin{itemize}
\item Definition (\texttt{./sec/symbols.tex}):
\lstset{language=[LaTeX]TeX,label= ,caption= ,captionpos=b,numbers=none}
\begin{lstlisting}
\newglossaryentry{omega}
{
    name={\ensuremath{\omega}},
    description={angular velocity},
    sort=omega,
    symbol={\ensuremath{\omega}},
    unit={\si{rad/s}}
}
\end{lstlisting}
\item Usage: this is \texttt{\textbackslash{}gls\{omega\}}.
\end{itemize}

\section{Listings}
\label{sec:org52673a7}
\begin{itemize}
\item Styling: Listings can be formatted using different styles, as presented in
\texttt{./sty/listing.sty} for any programming/markup language required.
\begin{itemize}
\item Example: C
\lstset{language=[LaTeX]TeX,label= ,caption= ,captionpos=b,numbers=none}
\begin{lstlisting}
\lstdefinestyle{customc}{
belowcaptionskip=1\baselineskip,
breaklines=true,
%frame=L,%lines, whole
xleftmargin=\parindent,
language=C,
showstringspaces=false,
basicstyle=\scriptsize\ttfamily,
keywordstyle=\bfseries\color{green!40!black},
commentstyle=\itshape\color{purple!40!black},
identifierstyle=\color{blue},
stringstyle=\color{orange},
numberstyle={\tiny},
numbers=left,
numberblanklines=false,
stepnumber=5,
backgroundcolor=\color{yellow!10}, 
frame=tlb
}
\end{lstlisting}
\end{itemize}
\item Usage: Listings can be inputted using the desired style as below. It includes
a caption, a label, a style, and a file path:
\lstset{language=[LaTeX]TeX,label= ,caption= ,captionpos=b,numbers=none}
\begin{lstlisting}
\lstinputlisting[language=C++, caption={Thread Serial Rx handler},label=lst:threadSerialRx,
style=customc]{./listing/threadSerialRx.cpp}%
\end{lstlisting}
\item Result:
\end{itemize}
\lstset{language=C,label= ,caption= ,captionpos=b,numbers=none,style=customc}
\begin{lstlisting}
UINT MMSLSDlg::ThreadSerialRx(LPVOID param)
{
    /* Wait for 1st connection to serial port: OnConnect */
    ::WaitForSingleObject( EvSerial.m_hObject , INFINITE); 
    tstring szData;
    CDemoEzdDlg *dlg = (CDemoEzdDlg *)param;

    while(1)
    ;

    return 0;
}
\end{lstlisting}
\section{PDF inclusion}
\label{sec:orgd6f8882}
\begin{itemize}
\item Include all pages from \texttt{anexo3-license}. Be careful of file path. It's
relative to the main file.
\lstset{language=[LaTeX]TeX,label= ,caption= ,captionpos=b,numbers=none}
\begin{lstlisting}
\includepdf[pages=-]{anexo3-license}
\end{lstlisting}
\end{itemize}
\section{Appendices}
\label{sec:org8624020}
Appendices can be added in the appropriate section (\texttt{./tex/Append/}):
\begin{itemize}
\item as text: with figures, tables, etc.
\item included as PDF
\end{itemize}
\end{document}