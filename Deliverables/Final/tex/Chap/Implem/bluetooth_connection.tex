\subsection{Bluetooth}%
\label{sec:bluetooth-implem}
%
Bluetooth is one of the (redundancy) communication technologies specified in the project analysis (section \ref{ch:analysis}), hence it is important to assure the data flow between the smartphone and the \gls{nvs} on another redundancy communication failure like Wi-Fi. 
%
\subsubsection{Bluetooth Connection Setup}
\label{sec:bluetooth-connection-setup}
Starting from the design and reusing previous Java code made on other course units (based on \cite{btchatsrc}), the Bluetooth setup was implemented on Android Studio and then deployed in a smartphone, the code is depicted in listing \ref{lst:bt-setup}.
%
Note that the forementioned code has \textbf{threads} to simulate parallel computing since the app should be able to send data to the paired device and receive data from it.
%
In lines 10, 11 and 12, are represented the three thread classes used for the Bluetooth connection setup. These classes inherit from the \textbf{Thread class} (\underline{extends Thread} - line 114), considering it must exist \textbf{concurrency} and \textbf{resource sharing} for the application to send and receive data simultaneously. 
%
The \textbf{AcceptThread} (line 10) it's related to the discovery of new connections since it plays an important role in the creation of a \textbf{BluetoothServerSocket} that allows two intended devices to find and accept each other as a part of the initial device inquiry prior to the connection stage. Line 11 represents the \textbf{ConnectThread} responsible for managing the connection between two devices. This thread starts as soon as the two devices accept each other and then proceeds to create a \textbf{RfcommSocket} and connect the devices (pair) when the user clicks on an unpaired device from the list presented. Finally, the \textbf{IOThread} assures the message exchange between devices by accessing the input and output streams of the \textbf{Bluetooth Socket}.
%
As stated earlier, it was necessary to make some includes in the Bluetooth app as well as virtualize some ports in order to interface the \gls{pc} virtual machine with the smartphone, this will be more thoroughly explained in the testing section (\ref{sec:bluetooth-phone-nvs}).
%
Finally, from the smartphone point of view, the protocol purpose was to transmit commands to control the vehicle by sending the phone accelerometers' values and receive its message warnings on screen.\\
%
\lstinputlisting[language=Java,caption={Code for bluetooth connection setup},label=lst:bt-setup,style=custom-java]{listing/btSetup.java}
%
%%% Local Variables:
%%% mode: latex
%%% TeX-master: "../../../dissertation"
%%% End:


