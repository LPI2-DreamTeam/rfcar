\subsection{Wi-Fi}%
\label{sec:wifi-implem}
%
Wi-Fi is the second communication technology adopted to guarantee effective control of the rover with the commands from the smartphone.
%
\subsubsection{Wi-Fi Connection Setup}
\label{sec:wifi-implem-connection}
%
The Wi-Fi connection setup code implementation is listed in \ref{lst:wifi-connection-setup}. As one would expect it follows a \textbf{client-server approach} represented in the mentioned listing by the classes \textbf{ServerClass} (line 218) and \textbf{ClientClass} (line 238). Note that both classes inherit from the \textbf{Thread} class since, once again, \textbf{concurrency} is a relevant factor. On one hand, the \textbf{ServerClass} it's responsible for creating a \textbf{ServerSocket} and specify the \textbf{port} that will be used in the connection. This type of protocol usually requires the determination of the \textbf{IP address} and \textbf{port number} that will be utilized. On the other hand, the \textbf{ClientClass} also has to specify the \textbf{port} used but also the forementioned \textbf{IP}. Due to the \gls{ide} and programming language option chosen in section \ref{sec:smartphone-implem}, one is now limited to use the predetermined IP protocol version preemptively defined (in this case \textbf{IPv6}) what might not be optimal. Both classes use another class that inherits from \textbf{Thread}, the \textbf{WifiConnectionManager} class, to handle the\textbf{ Wi-Fi input and output streams} of the \textbf{server and client sockets}, if the device is host or client, respectively. With this \textbf{WifiConnectionManager} class, one can manage the message exchange between the devices and assure its simultaneity.\\
%
\lstinputlisting[language=Java,caption={Code for Wi-fi connection setup},label=lst:wifi-connection-setup,style=custom-java]{listing/wifiConnectionImplem.java}
%
\subsubsection{Wi-Fi Video Feed}
\label{sec:wifi-implem-video}
%
One important user feature defined in the design of the smartphone application in section \ref{sec:smartphone-design} was the capability of watching the \gls{rvvs}' live camera feed within the app. The applied solution is based on the \textbf{Youtube Android Player \gls{api}} \cite{yt_api} as it enables a more \textbf{reliable video stream} whilst ensuring \textbf{project off-load}. Resorting to the latter, one can also tackle the fullscreen video display which was a big concern in terms of user-friendliness. Nevertheless, one can still claim that a framewise transmission approach via Wi-Fi is even more fail-safe and secure since one doesn't need to rely on an external entity for video communication.
Despite this, the framewise approach was discarded to ensure deadline meeting, hence its implementation required some prior subject experience or further research. The implementation of the abovementioned feature is depicted in listing \ref{lst:video-view}.\\
%
\lstinputlisting[language=Java,caption={Code for video feed view within the application},label=lst:video-view,style=custom-java]{listing/videoFeed.java}
%
