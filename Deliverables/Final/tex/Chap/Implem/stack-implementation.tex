%intro
As forementioned, the initial approach consisted of the three-way interaction depicted in the RFCAR deployment diagram, in section \ref{sec:hw-sw-mapping} - figure \ref{fig:deployment-diag}. However, due to the extraordinary circumstances the need to virtualize the navigation and remote vision subsystems rose. Specifically, both were simulated on a virtual machine with an \textbf{18.04 Lubuntu image}, a Ubuntu-based lightweight Linux distribution. This implies that the High-level hardware abstraction layer's implementation should be done using Linux-specific APIs and libraries. As such, in an effort to meet the established deadlines the liberty of using C++ and its \gls{stl} was taken. 
%
\subsubsection{IO: Input/Output Package}
%
\subsubsection{COM: Communications Package}
%
\subsubsection{OS: Scheduler Package}
%
\subsubsection{MEM: Memory Structures Package}
%
\subsubsection{CLK: Timing Package}
%
\subsubsection{APP: Main Application Package}
%