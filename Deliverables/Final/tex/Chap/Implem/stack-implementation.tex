%intro
The chosen language of implementation was C++ due to its low-level nature and backwards compatibility with the C programming language, which is what most Board Support Packages (BSP's) are written in. The fact that it was chosen over C, though, is mostly due to its object-oriented nature, that translates very easily from the package and entity-oriented design, UML class diagrams and package diagrams and makes for cleaner, more easily maintainable code. To mitigate the effects of the C++ runtime on performance, inheritance was thoroughly avoided, class templates and template specializations being the alternative to this method, as seen in the IO::Entity and COM::LL subpackages. This allows us to trade storage footprint for better runtime performance. Furthermore, the use of C-style types was preferred whenever possible in the higher-level layers, practicality
As forementioned, the initial approach consisted of the three-way interaction depicted in the RFCAR deployment diagram, in section \ref{sec:hw-sw-mapping} - figure \ref{fig:deployment-diag}. However, due to the extraordinary circumstances the need to virtualize the navigation and remote vision subsystems rose. Specifically, both were simulated on a virtual machine with an \textbf{18.04 Lubuntu image}, a Ubuntu-based lightweight Linux distribution. This implies that the High-level Hardware Abstraction layer's implementation should be done using Linux-specific APIs and libraries. As such, in an effort to meet the established deadlines the liberty of using C++'s \gls{stl} was taken. This stability and convenience was also a strong motivation for choosing C++ as the language of implementation.
%
\subsubsection{IO: Input/Output Package}
As foresaid in section \ref{sec:io-package}, the IO package includes the IO Entity and IO Link subpackages. The IO\_Entity header file represented in listing \ref{lst:io-entity-interface}, includes a declaration of a generic template (within IO namespace) that allows the specialization of GPIO objects (belonging to the IO Link package) in physical entities like a motor or an infrared sensor, in this case. Each specialization of an Entity object requires the definition of the targeted constructor. There, one can decide which type the entity is going to take and its position within the rover model whilst configuring the entity attending to its GPIO requirements. This is done through preemptively defined GPIO-targeted modes in \ref{lst:io-GPIO-interface}.\par
%
The second subpackage, IO Link, is listed in \ref{lst:io-GPIO-interface} and \ref{lst:io-GPIO-source}. This generic package interacts directly with the virtualization of the machine itself (sensors and actuators) through timed input and output in binary files while also defining targeted GPIO modes for certain entities. The IO interface in listing \ref{lst:io-interface} allows the definition of transversal structures and typedefs for the IO-related packages.
%
%IO_GPIO.cpp
\lstinputlisting[language=C++, caption={IO\_GPIO Source},label=lst:io-GPIO-source,
style=customc]{./listing/IO_GPIO.cpp}
%
%
\subsubsection{COM: Communications Package}
\paragraph{COM::LL}
The COM::LL package possesses 3 templates, depending on which entities are communicating (client or server) and which protocol is being used to communicate (serial or Bluetooth). The main difference between the three lies in how they communicate, the serial protocol uses local sockets (AF\_LOCAL) to be able to communicate with the Remote Vision Subsystem in the same machine whilst Bluetooth uses a connection to a serial port that is associated with the Bluetooth adapter of the host device and any communication via object of this class is done through this port.
The package provides means of communication via reading and writing on buffers and is implemented with several error verifications and feedback messages indicating either the reason for errors or that the operation was a success. The error verifying during the creation of the of the objects makes sure of the existence of the requested port, followed by ensuring that no two objects access the same port. Each time one of the available means is used for communicating, it is always verified if the client-server connection has already been established and if not it doesn't permit sending or reading messages.
%
\lstinputlisting[language=C++, caption={COM\_LL},label=lst:com-ll-source,
style=customc]{./listing/COM_LL.cpp}
%
\subsubsection{OS: Scheduler Package}
\paragraph{OS::Mutex}
Has the purpose of managing concurrent access in a way that avoids errors, as previously mentioned in section~\ref{sec:osPack}.

\lstinputlisting[language=C++, caption={OS\_Mutex},label=lst:mem-mutex-source,
style=customc]{./listing/OS_Mutex.cpp}

\paragraph{OS::SharedMemory}
Makes use of the std::list container for storing the thread whitelist, for it requires efficient methods for insertion and removal of elements.

\lstinputlisting[language=C++, caption={OS\_SharedMemory},label=lst:mem-sharedmemory-source,
style=customc]{./listing/OS_SharedMemory.hpp}

\paragraph{OS::Thread}
The execution of the method is controlled by a call to the \texttt{run()} method which gives the start signal to the main method guaranteeing that whichever thread called the \texttt{run()} method is the parent thread. It permits a precise control of the total time of the thread whilst making sure that the only thread that can control it is itself.
%
\lstinputlisting[language=C++, caption={OS\_Thread},label=lst:mem-thread-source,
style=customc]{./listing/OS_Thread.cpp}

\paragraph{OS::Notification}

\lstinputlisting[language=C++, caption={OS\_Notification},label=lst:mem-notification-source,
style=customc]{./listing/OS_Notification.cpp}

\subsubsection{MEM: Memory Structures Package}

Both lists provide efficient methods for insertion and removal of elements and completely clearing the list, all of these operations with O(1) complexity.

\paragraph{MEM::CircularList}
It is implemented with recourse to a C-style array to optimize for use as a single-byte data container (e.g. strings of characters in message buffers).

\lstinputlisting[language=C++, caption={MEM\_CircularList},label=lst:mem-cl-source,
style=customc]{./listing/MEM_CircularList.hpp}

\paragraph{MEM::LinkedList}
Implemented with a std::list from the STL of C++ due to the well devised methods made available by this type of container, ensuring a smooth and practical implementation. Also std::list is in and of itself the STL implementation of a linked list, guaranteeing the expected algorithm complexity of a true linked list.

\lstinputlisting[language=C++, caption={MEM\_LinkedList},label=lst:mem-ll-source,
style=customc]{./listing/MEM_LinkedList.hpp}
%
\subsubsection{CLK: Timing Package}
The \textbf{CLK} package is comprised by only the \textbf{Timer} module and convenient type definitions. It is very useful for dealing with the periodicity of each thread and provides a means for setting up repeated timed delays and executing a specific routine automatically when each delay ends. It provides an interface for waiting for the end of the timed delay and/or the execution of the specified routine and autonomously notifies all waiting objects of these events.
It is mainly used by the \textbf{IO::GPIO} class for timing conversions but it can as easily be used by higher-level classes due to its \textbf{versatile interface} and \textbf{general-purpose} nature.
%
%IO_GPIO.cpp
\lstinputlisting[language=C++, caption={Timer},label=lst:Timer,
style=customc]{./listing/CLK_Timer.cpp}
%
\subsubsection{APP: Main Application Package}
%