%\section{Smartphone}
%\label{sec:smartphone-implem}
The mobile \gls{os} chosen for this project was \textbf{Android}. Usually, android apps can be implemented using Kotlin, Java and C++. For the \gls{rfcar} project, the language chosen was \textbf{Java} due to the knowledge gained in prior course years where the need to implement android-targeted applications rose. Additionally, the code was conceived using the \textbf{Android Studio} \gls{ide}. One must notice that despite the user-friendliness of the development environment through context-sensitive guidelines and code suggestions, this language and \gls{ide} are not the best in terms of full system control due to the multiple abstraction layers preemptively defined. Therefore, one might wonder why the C++ route with a cross-platform framework like QT wasn't selected. Multiple options were taken into account at this stage and its a fact that the one that ended up being selected might not deliver as much implementation freedom as the second option forementioned. However, it allowed deadline fulfilment and code reuse notwithstanding the additional effort put into delving deeper within some contexts.
%
\subsection{Sensor Interaction}%
\label{sec:accelerometer-access}
%
The smartphone has a built-in MEMS accelerometer, this means that at micro-level it can measure acceleration values through capacitance changes in multiple capacitors as a result of its internal assembly (calibration mass and spring contacts) displacement. With at least a MEMS system in each plane (x,y,z), one can measure the acceleration per axis. 
\subsubsection{Sensor Data Retrieval}
\label{sec:accelerometer-data}
%
The code in listing ~\ref{lst:get_accelerometer_vals} (based on \cite{androiddevsensors}) represents the retrieval of the linear acceleration for each plane.
In the first place, the definition of the sensor type is crucial to access its values (line 9). \textbf{SensorManager} grants access to the sensors of the android device (line 7). Following, one must create a listener that checks the sensors values at a determined sampling frequency using the SensorManager's method \textbf{registerListener} (line 10). Upon doing so, one overwrites the \textbf{onSensorChanged} method (line 15) so the pretended variables that hold the values of the sensor can only be updated on smartphone movement.
An acceleration sensor measures the acceleration applied to the device but regarding the force of gravity. For this reason, the values retrieved do not represent the linear accelerations for each plane. To resolve this problem, a low pass filter can be used to isolate the force of gravity in each axis (line 28) and then remove its contibution from the acceleration values (line 33).\\
%
\lstinputlisting[language=Java,caption={Accelerometer data retrieval code},label=lst:get_accelerometer_vals,style=custom-java]{listing/accelerometerAccess.java}
%
As the control module uses both wheels tilt angle and tension applied to the motor as input, the interface with the smartphone's accelerometer wasn't enough. To generate the angles necessary and vary the tension accordingly one needs to access the phone's rotation sensors. This code is implemented in listing \ref{lst:get_rot_vals}. As the interface with the accelerometer, this latter interface needs a \textbf{SensorManager} and the creation of a listener with the \textbf{registerListener} method. Some calibrations were crucial to ensure the calculated values of the angles matched the initial smartphone position chosen.\\
%
\lstinputlisting[language=Java,caption={Rotation sensor data retrieval code},label=lst:get_rot_vals,style=custom-java]{listing/rotAccess.java}
%
\subsubsection{Applying Sensor Data}
\label{sec:using-accelerometer-data}
%
The project requires that the acceleration values obtained from the sensors are applied to make the vehicle move in the intended direction with the expected speed. 
%
To implement the code referent to this sub-subsection, one must pay close attention to the axis orientation in a common device, represented in figure \ref{fig:axis-smartphone}.
%
\begin{figure}[!h]
\centering
\includegraphics[width=0.85\textwidth]{img/smartphone_axis.png}
\caption{\label{fig:axis-smartphone}Axis orientation in a smartphone}
\end{figure}
%
In order to test this concept, the code presented in \ref{sec:accelerometer-data} that refers to the accelerometer was added to another project that allowed ball movement based on the accelerometer values. It must be noticed that the z acceleration value it's not relevant to the ball movement (from line 59 to 67) since only \textbf{roll} and \textbf{pitch} affect a 2D (bidimensional) object and the smartphone height relative to the ground doesn't, as one should expect by analysing figure \ref{fig:axis-smartphone}.
%
This code in listing \ref{lst:ball_mov} representing subsection \ref{sec:accelerometer-access} will be tested in \ref{sec:accelerometer-using-data-test}.\\
%
\lstinputlisting[language=Java,caption={Code for ball movement based on accelerometer data},label=lst:ball_mov,style=custom-java]{listing/ballMovement.java}
%
%
\subsection{Bluetooth Connection}%
\label{sec:bluetooth-connection}
As aforementioned earlier in this dissertation, an initial approach consisted of an android that interacted with a RaspberryPi and the STM board.
%
However, due to the extraordinary conditions, one can only rely on the virtualization of the latter two for the perpetuation of the following project.
Specifically, the STM board was simulated on a virtual machine with an 18.04 Lubuntu image, a lightweight Linux distribution based on Ubuntu that provides more freedom, simplicity and compatibility as opposed to other operating systems like Windows 10.
%
Starting from the design and reusing previous Java code made on other course units, the Bluetooth setup was implemented on Android Studio and then deployed in a smartphone, the code is depicted in Fig. xx.
%
Note that the following code has threads to simulate parallelism computing since the app should be able to send data while also listening to the paired device for receiving data as well.
%
Notice also that it was necessary to make some includes in the Bluetooth app as well as virtualize some ports to interface the virtual machine with the smartphone connected to the pc, more thoroughly explained in the testing section.
%
Finally, from the smartphone point of view, the protocol purpose was to transmit commands to control the vehicle, via sending phone accelerometers' values and receive its message warnings on screen. 
%
So for that reason, an app was made to access the accelerometers. Those values serve as an input, changing a ball direction and velocity controlled this way by the tilting angle of the phone, code shown in Fig. ss.
%%% Local Variables:
%%% mode: latex
%%% TeX-master: "../../../dissertation"
%%% End:



%