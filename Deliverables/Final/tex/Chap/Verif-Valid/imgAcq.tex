\section{Image Acquisition}%
\label{sec:img-acq-verif}
In this section is verified the compliance between foreseen (see
Section~\ref{sec:image-acquisit}) and actual specifications (see
Section~\ref{sec:img-acq-rvvs-test}).

As can be seen in this latter section, unfortunately, the available webcam is
very old, supporting only the \texttt{uyvy422} format (luminance +
chrominance). This prevents the straightforward video capture and its streaming
to a remote streaming platform (e.g. \texttt{Youtube}) where it's ubiquitously
available through a shareable link. A more convenient format would be the
MJPEG.

Concerning the framerate, it can be that the webcam supports 30 fps
only. However, it's actual value in dependent on system's load, but, as video
capture was not possible, consequently, it is left undetermined.

For the same reason, and although several resolutions are available (640x320,
352x288, 340x240, 176x144, and 160x120), it was only possible to test out image
acquisition at these resolutions, with success, but lacking the video capture
tests, which is not critical, as it does not depend on dynamic conditions, only
on a statically defined capability of the webcam.
%%% Local Variables:
%%% mode: latex
%%% TeX-master: "../../../dissertation"
%%% End:
