	% CHAPTER - Conclusion/Future Work --------------
\chapter{Conclusion}%
\label{ch:conclusion}
In this chapter are outlined the conclusions and the prospect for future work
regarding the \gls{rfcar} project.
%
\section{Conclusions}%
\label{ch:conclusion-concls}
The realization of this project went through several phases, following the
waterfall methodology and top-down approach with two process iterations, modular
and integrated.

In the analysis stage, the foreseen specifications were listed, as well as the
envisioned tests for verification and validation of the product. Additionally, a
preliminary design was sketched as a possible viable solution.

In the design phase, the considerations drawn in the analysis, combined with the
initial design, were used to conceptualize a viable solution for the product
materialization. The system as decomposed into smaller subsystems that was
conducted by specialized teams, namely: Hardware control, smartphone, STM32,
camera and Raspberry Pi.

The implementation phase started with a modular implementation of all the
referred smaller subsystems and consecutive test. This phase always want hand to
hand with the design since the functionalities didn't always work at
first. After the modular implementation came te integrated implementation where
all the smaller subsystem were combined in order to form the final product. Even
though some tests were successful in the integrated implementation, others were
not possible to accomplish.

After the implementation and respective tests, came the verification phase where
the focus test the the foreseen specifications.

Finally, came the phase of product validation in which an external agent tested the product interface according to the instructions provided.
The documentation was always a strong pillar in all the referred phases, and as
such, was always updated every time any progress was made.

From a critical point of view, this project suffered a large restriction due the
current situation the world is in, which hindered the interaction between the
members of the group making it difficult to manufacture a prototype and to
integrate the functionalities modularly implemented. As such, the base of the
project was model-based design and simulation, which is a must-have tool for
agile and cost-effective development of products.

Despite the obstacles, this project can be considered a accomplishment due to
the success of the android interface, multi platform communications, camera and
control algorithm within a restricted timeline and budget, supported by the
methodology used and the model-based design framework.
%
\section{Prospect for Future Work}%
\label{ch:conclusion-future-work}
In the foreseeable future, an actual prototype could be implemented using the
developed control algorithms, communications, camera and interface, as was the
initial plan.

In order to make the system better, it could be used a different camera, with
better frame rate, resolution and range and more image formats. Also, redundancy could be added to communication making the system more stable and less fault-prone. 

To create a better environment for the user, some upgrades could be made such
as, finish video feed interface, add \gls{gps} tracking to assess the position of the car at all times, and more rigorous odometry that does not
force the vehicle's stoppage when facing an obstacle, but allows it to circumvent it.

Finally, the prototype could be tested in different real environments, as envisioned.
%%% Local Variables:
%%% mode: latex
%%% TeX-master: "../../../dissertation"
%%% End:
