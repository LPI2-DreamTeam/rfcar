% Created 2020-07-16 Thu 09:09
% Intended LaTeX compiler: pdflatex
\documentclass[11pt]{article}
\usepackage[utf8]{inputenc}
\usepackage[T1]{fontenc}
\usepackage{graphicx}
\usepackage{grffile}
\usepackage{longtable}
\usepackage{wrapfig}
\usepackage{rotating}
\usepackage[normalem]{ulem}
\usepackage{amsmath}
\usepackage{textcomp}
\usepackage{amssymb}
\usepackage{capt-of}
\usepackage{hyperref}
\usepackage{color}
\usepackage{listings}
\author{Hugo Carvalho A85156, Hugo Ferreira A80665, João Faria A85632, João de Carvalho A83564, José Mendes A85951, José Pires A50178\thanks{a50178@alunos.uminho.pt}}
\date{\textit{<2020-03-10 Tue>}}
\title{LPI2 - Project Integrator: RFCAR}
\hypersetup{
 pdfauthor={Hugo Carvalho A85156, Hugo Ferreira A80665, João Faria A85632, João de Carvalho A83564, José Mendes A85951, José Pires A50178},
 pdftitle={LPI2 - Project Integrator: RFCAR},
 pdfkeywords={},
 pdfsubject={},
 pdfcreator={Emacs 26.3 (Org mode 9.3.6)}, 
 pdflang={English}}
\begin{document}

\maketitle
\tableofcontents


\section{Gestão do Projecto}
\label{sec:org6b887bc}
\subsection{Filesystem: Directórios}
\label{sec:org675f262}
\textbf{Estrutura do filesystem}: Cada directório deverá conter um ficheiro \texttt{rd} de
 base; todos os outros poderão ter outro nome, mas este será convencionado.
\begin{itemize}
\item \href{rd-PI.org}{rd-PI}: contém a informação genérica essencial acerca do projecto.
\item \href{sec/img}{./sec/img/}: contém as imagens relativas aos ficheiros de documentação. Cada
subdirectório deverá conter uma pasta deste tipo.
\item \href{ProjManag}{./ProjManag/}: contém os documentos essenciais à gestão do projecto: diagrama
de Gantt, horas do projecto, etc.
\item \href{Proj}{./Proj/}: contém a informação relacionada com o projecto dividida pelas várias
iterações. Os documentos gerados neste directório servirão de base à escrita
do relatório final, sendo, por isso, mais informais e destinando-se a manter
todos os grupos de trabalho actualizados.
\begin{itemize}
\item \href{Pesquisa}{./Proj/Iterations/<x>/Research/}: Deve conter todo o \emph{knowledge base} para o
projecto. É a primeira  etapa do mesmo, mas deverá ser incrementado sempre
que necessário. Deve conter o conceito do produto a desenvolver, quais os
objectivos, a utilidade, etc. \textbf{x} representa o número da iteração.
\item \href{Analise}{./Proj/Iterations/<x>/Analysis/}: Contém os resultados da Análise da pesquisa
inicial. Deve resultar na elaboração dos requisitos e restrições ao
projecto/produto. Deverá ter uma base quantitativa para iniciar o design,
constituindo o conjunto de especificações preliminares do produto:
\begin{itemize}
\item p. ex.: velocidade deverá ser, no máximo, até 2 m/s.
\end{itemize}
\item \href{Design}{./Proj/Iterations/<x>/Design/}: contém o desenvolvimento de uma, ou mais,
solução(ões) para o produto. Deverá incluir toda a parte de modelação da solução, p. ex.:
\begin{itemize}
\item Design do sistema de controlo: analiticamente e recorrendo a simulação em
scilab
\item Design da sensorização do sistema: design dos circuitos e respectiva
simulação
\item Design da parte de potência: alimentação, actuação dos motores e respectiva
simulação
\item Design da parte mecânica: design do tapete
\end{itemize}
\item \href{Implem}{./Proj/Iterations/<x>/Implem/}: contém a implementação do sistema: 
\begin{itemize}
\item desenhos do controlador, sistemas de medição e de potência prontos para
testes;
\item desenhos das peças mecânicas a produzir e/ou adquirir
\item Lista de materiais
\begin{itemize}
\item Na fase final do projeto, deverá incluir os PCBs
\end{itemize}
\end{itemize}
\item \href{Testes}{./Proj/Iterations/<x>/Tests}: contém todos os testes realizados ao sistema e
aos vários protótipos. Consideram-se testes apenas os que são realizados
sobre qualquer componente ou protótipo físico.
\end{itemize}
\item \href{Deliverables}{./Deliverables/}: contém todos os elementos a entregar acerca do projecto:
relatórios, desenhos finais, apresentações, etc.
\begin{itemize}
\item \href{Deliverables/prod-concept.org}{prod-concept}: conceito de produto a entregar contendo (até
\textit{<2019-11-05 Tue>}):
\begin{itemize}
\item constituição do grupo
\item Orientador
\item Conceito do produto a desenvolver: \textbf{Kit didáctico dum sistema de controlo
de carga num tapete rolante}
\end{itemize}
\end{itemize}
\item \href{Doc}{./Doc/}: contém a info necessária para reproduzir o projecto.
\item \href{readme.org}{./writing/}: ficheiros relativos à escrita do relatório final de forma
incremental.
\item \href{SW}{./SW/}: software desenvolvido para o projecto decomposto pelas várias
vertentes.
\item \href{HW}{./HW/}: HW desenvolvido para o projecto decomposto pelas várias vertentes.
\end{itemize}
\subsection{Horas do Projecto}
\label{sec:org6fc3a8d}
Cada elemento do grupo deverá adicionar o nr. de horas dedicado ao projecto
no ficheiro \href{ProjManag/ProjHours.xlsx}{ProjHours.xlsx}.Basta duplicar a folha inicial.
\subsection{Contactos}
\label{sec:org35be522}
Aqui estão os contactos de todos os elementos do grupo. Sintam-se à vontade para editar.
\begin{center}
\begin{tabular}{lrll}
Nome & Nr. Aluno & Email & Tlm\\
\hline
Nuno Rodrigues & 85207 & nunorodrigues0707@gmail.com & \\
Hugo Carvalho & 85156 & hugo.mitab@gmail.com & \\
Hugo Ferreira & 80665 & hugunited11@gmail.com & \\
João Faria & 85632 & joaofaria99@gmail.com & \\
João Carvalho & 83564 & jafpcarvalho44@gmail.com & \\
José Mendes & 85951 & josepr.mendes@gmail.com & \\
José Pires & 50178 & a50178@alunos.uminho.pt & 911 901290\\
\end{tabular}
\end{center}

\section{Tools}
\label{sec:org43134b7}
\subsection{Team conversation}
\label{sec:org10b807a}
A ferramenta \href{https://twist.com}{Twist} foi escolhida por fornecer agrupamento de conversas em
tópicos
\subsection{Diagrama de Gantt}
\label{sec:orgee1b3c6}
Será usado o \href{https://www.gantter.com/}{Gantter} para elaboração do diagrama de Gantt requirido pela
etapa 0. Este diagrama deverá ser o mais minucioso possível e será utilizado
para rastreamento e gestão do projecto.
\begin{itemize}
\item Dado que é gratuito durante 30 dias, será necessário criar contas adicionais
de email.
\item Permite exportar e importar ficheiros do tipo \texttt{.gantter} (ver \href{sec/examples/PL2.gantter}{exemplo}).
\end{itemize}

O mapa de Gantt original deverá ser preservado, como constando da etapa 1, mas
deverá ser actualizado ao longo do projecto.
\subsection{Get Things Done}
\label{sec:org745fc98}
O diagrama de Gantt deverá ser usado numa macro-escala para gestão do projecto.
Contudo, muitas vezes estas são ainda de granularidade elevada. Assim, cada
elemento, ou sub-grupo tem liberdade de utilizar uma ferramenta para definição
de tarefas mais pequenas, o que poderá ajudar à produtividade.
\begin{itemize}
\item Estas subtarefas devem enquadrar-se de algum modo no diagrama de Gantt.
\item Uma ferramenta deste tipo é o \href{https://trello.com/}{Trello}.
\end{itemize}

Cada elemento do grupo deverá adicionar o nr. de horas dedicado ao projecto
no ficheiro \href{ProjManag/ProjHours.xlsx}{ProjHours.xlsx}.Basta duplicar a folha inicial.
\subsection{Documentation + Tracking Project}
\label{sec:orgac9a1b9}
Existe um grande fluxo de informação associado ao projeto. Para gerir melhor
essa informação cada grupo de trabalho deverá geral a documentação relativa aos
seus tópicos, nomeadamente:
\begin{itemize}
\item \emph{código}: usando o doxygen
\item \emph{documentação genérica}: usando o \LaTeX{}
\item \emph{gestão de projecto}: usando o Trello para adicionar tarefas e atribuí-las aos
diferentes membros da equipa.
\end{itemize}

\subsubsection{Workflow}
\label{sec:org9de74ce}
Para a gestão da informação relacionada com o projecto será necessário adoptar
algumas normas e convenções para utilização adequada e conveniente do repositório.
\begin{enumerate}
\item \textbf{Filesystem}: Será criado um \emph{filesystem}, cuja raiz é o directório actual e
cujas paths devem ser relativas. Para armazenamento do \emph{filesystem} será
usada o github. Para visualizar e navegar pelo conteúdo basta replicar o
\emph{filesystem} em qq directório. O \emph{filesystem} será navegável através dos
links indicados neste readme file na secção \hyperref[sec:org675f262]{Directórios} deste documento.
\item \textbf{Estrutura}:
\begin{enumerate}
\item Existirá um ficheiro readme \texttt{rd.tex} por cada directório que deve conter a
informação essencial a esse tópico e permite a navegação para os restantes
subtópicos através de links.
\item Cada pasta deve conter um directório \texttt{./sec/img/} aonde serão armazenadas
as imagens relevantes para o tópico.
\item Este ficheiro poderá ser compilado usando o \LaTeX{}, gerando-se o respectivo
PDF que estará também disponível no mesmo directório.
\end{enumerate}
\end{enumerate}

\textbf{Workflow}:
\begin{enumerate}
\item \textbf{Projecto}:
\begin{itemize}
\item As tarefas deverão ser adicionadas a cartões do Trello e atribuídas aos
diferentes elementos da equipa, com a data prevista de entrega.
\item Cada cartão deverá ter uma data prevista de entrega, não sendo recomendado
misturar tarefas para diferentes datas.
\item Na vista \emph{Calendar} é possível visualizar os cartões do projecto.
\end{itemize}
\item \textbf{Código}:
\begin{itemize}
\item O código deverá ser adicionado ao directório específico pelo grupo de
trabalho indicado que deverá supervisionar a gestão deste, através dos
"pushs" e "merges".
\item Deverá ser incluído um Doxyfile para geração da documentação quando
requisitada. Não será necessário atualizar toda a documentação para o
repositório já que o Doxygen poderá fazê-lo mediante um ficheiro de
documentação válido e os ficheiros codificados com as tags correctas.
\end{itemize}
\item \textbf{Doc}:
\begin{itemize}
\item Criar os ficheiros \texttt{.tex} desejados, e.g., \texttt{rd.tex}.
\item Compilar o ficheiro \texttt{.tex} e gerar o respectivo PDF.
\end{itemize}
\end{enumerate}

\subsection{Git}
\label{sec:org2f8e126}
Git is a version control tool, providing easy management of a distributed
document source between the several elements of the team.

\subsubsection{Workflow}
\label{sec:orgc1f6901}
\begin{enumerate}
\item Basic
\label{sec:org7bfd8b9}
\begin{enumerate}
\item \textbf{Clone} the project
\item The administrator will add you as a colaborator, giving you direct push
permissions.
\item \textbf{Pull} the last changes from remote repository before starting to work in
the project and just before pushing changes to remote repository:
\begin{itemize}
\item using:  \texttt{git pull <branchname>}
\item this avoids most conflicts.
\end{itemize}
\item \textbf{Modify} the documents required, keeping it limited to a narrow scope: e.g.,
fix a bug, add a functionality, etc. Large modifications should be avoided.
\item \textbf{Stage} the changes: add them to tracking
\item \textbf{Commit} (or discard) the changes: if commited the changes will override the
previous files in the local repository.
\item \textbf{Push} the changes: send the local repository changes to the remote
repository (hosted on github) for update, using \texttt{git push <remoteName>}
\begin{itemize}
\item Conflicts can be detected if someone has modified:
\begin{itemize}
\item the remote repository (some files, but not the ones you've worked in):
for this, one can just pull changes (see 2)
\item the file(s) you've worked in: in that case, you have to pull the changes
(see 2), modify the files with your changes, and commit again.
\end{itemize}
\end{itemize}
\end{enumerate}
\item Advanced
\label{sec:org651005b}
The advanced workflow increments the basic one, by isolating the core source of
the project (current version fully operational) from the addition of new
functionalities or from code refinement. 

For this purpose, a branch should be created with a descriptive name from master
remote (preferably) and the functionality added.

\textbf{Workflow (advanced}:
\begin{enumerate}
\item \textbf{Branch} from master branch remote: \texttt{git branch <branchName> <remoteName>}
\item \textbf{Modify} the required files
\item \textbf{merge} the branch again to master remote, effectively adding the
functionalities to the core.: \texttt{git merge <branchName> <remoteName>}
\end{enumerate}
\end{enumerate}

\subsubsection{Commits' policy}
\label{sec:orgdeaac38}
The commits made first to local repository and then to the remote repository
should contain relevant information about the changes/deletions being made,
namely the reason behind them and what has been changed.

For this purpose, a set of thumb rules and tags are used to enforce this policy
as follows:
\begin{itemize}
\item \texttt{ADD: <what> to <where> and <why>}: addition of a new file, a new
functionality, a new snippet of code
\item \texttt{FIX: <what>, <where> and <why>}: fixing a bug, duplicate files, etc. fall in
this category
\item \texttt{UPDT: <what>, <where> and <why>}: updating the contents of a file, the file
structure, etc., follow in this category
\item \texttt{RM: <what> to <where> and <why>}: removing of a file
\end{itemize}

\textbf{Example}: \texttt{ADD: Git Workflow and Commit Policies to add consistency}
\section{Project}
\label{sec:org6085193}
\subsection{RFCAR - Radio Frequency Camera Assisted Rover}
\label{sec:org6b47175}
\subsubsection{Motivation}
\label{sec:org5cf0d10}
This project is being developed in the scope of the integrator project of
LPI2. It aims to develop skills in the software engineering area and digital
design, wireless communication protocols, odometric vehicle navigation, etc. 

\subsubsection{Description}
\label{sec:orgeca8efe}
  The project consists of a remote controlled car used to assist exploration
  and maintenance domains. For this purpose, the vehicle should contain a
  remotely operated camera feeding back video to the user. Additionally, the
  vehicle must contain odometric sensors to assist in driving and prevent
  crashes when user is not in control, e.g., when connection is lost.
The vehicle can be used for exploration of unaccessible areas to human operators
  like fluid pipelines and other hazardous sites.
\subsubsection{Technologies}
\label{sec:org5114a76}
\begin{itemize}
\item STM32: vehicle's low level control
\item Raspberry Pi: camera interface
\item Android: Human Machine Interface
\item Others (yet to be defined):
\begin{itemize}
\item Wi-Fi
\item Bluetooth
\item GPS
\item GPRS
\end{itemize}
\end{itemize}
\subsection{Workgroups}
\label{sec:org5a6642b}
\begin{enumerate}
\item \textbf{HW Control}: Nuno (50\%), Alex (50\%), Hugo F. (50\%)
\begin{itemize}
\item Car chassis
\item odometric sensors: ultrasonic + InfraRed
\item motors
\item buzzer
\item lights
\item battery
\item GPS
\end{itemize}
\item \textbf{Smartphone}: HMI - Hugo C. (80\%), João Faria (100\%), Zé Mendes (50\%)
\begin{itemize}
\item Accelerometers
\item GPS
\item Wi-Fi/ RF
\item GPRS
\item Display
\end{itemize}
\item \textbf{STM32}: Low-level SW layer control: Zé Mendes (50\%), Nuno (50\%), Alex (50\%),
Hugo F. (50\%),
\begin{itemize}
\item Control of car's HW
\item USART interface with HW controlling a camera
\end{itemize}
\item \textbf{Camera + control HW}: e.g. Raspberry Pi - Zé Pires(100\%), Hugo C. (20\%)
\begin{itemize}
\item Raspberry Pi Zero: runs Linux OS
\item Camera
\end{itemize}
\item \textbf{PCBs}: Hugo Carvalho, Hugo Ferreira
\end{enumerate}
\subsection{Meetings}
\label{sec:orgaefaf80}
\subsubsection{2 --- Planeamento}
\label{sec:orgca11fd1}
\textit{<2019-11-12 Tue 14:15>}

\begin{enumerate}
\item Tópicos
\label{sec:org108efd5}
\begin{enumerate}
\item Ferramentas de gestão do projecto: Typora, Pandoc, Excel (horas)
\item Análise da pesquisa preliminar sobre tapetes rolantes: treadmills
\item Definição dos componentes básicos para a concepção duma treadmill
\begin{itemize}
\item \textbf{Mecânica}: correia (belt), rolos (eixos)
\item \textbf{Electrónica}:
\begin{itemize}
\item Alimentação:
\begin{itemize}
\item Baterias vs Fontes de Alimentação comutadas
\end{itemize}
\end{itemize}
\item \textbf{Actuação}: motor DC e circuito de potência
\item \textbf{Controlo}: controlador analógico
\item \textbf{Sensores}: sensor de corrente para inferir carga a que o motor está
sujeito
\end{itemize}
\item Planeamento preliminar (dependente da aprovação inicial do tutor): Divisão do
projecto em várias etapas:
\begin{enumerate}
\item \textbf{Pesquisa} --- \emph{State of the art of treadmills}:
\begin{itemize}
\item Princípio de funcionamento
\item Identificação dos principais componentes do sistema e as suas
características
\end{itemize}
\item \textbf{Análise}: Contém os resultados da Análise da pesquisa inicial. Deve
resultar na elaboração dos requisitos e restrições ao projecto/produto,
Deverá ter uma base quantitativa para iniciar o design, constituindo o
conjunto de especificações preliminares do produto: - p. ex.: velocidade
deverá ser, no máximo, até 2 m/s.
\begin{enumerate}
\item Iteração 1: 3 dias \textit{<2019-11-14 Thu>}
\item Iteração 2: 3 dias
\end{enumerate}
\item \href{Design}{./Design/}: Pode ser dividido em \textbf{design conceptual} e \textbf{design da
solução}. 
\begin{itemize}
\item No design conceptual, são identificadas as várias soluções possíveis
para o problema, sendo quantificada a sua relevância para o projecto
através duma escala, inserida numa matriz de avaliação, p.ex., \href{https://en.wikipedia.org/wiki/Quality\_function\_deployment}{QFD}.
\item Design da solução: contém o desenvolvimento da solução identificada
para o produto. Deverá incluir toda a parte de modelação da solução,
p. ex.:
\begin{itemize}
\item Design do sistema de controlo: analiticamente e recorrendo a
simulação em scilab
\item Design da sensorização do sistema: design dos circuitos e respectiva
simulação
\item Design da parte de potência: alimentação, actuação dos motores e
respectiva simulação
\item Design da parte mecânica: design do tapete
\end{itemize}
\item Iteração 1: 2 semanas
\item Iteração 2: 1 semana
\end{itemize}
\item \href{Implem}{./Implem/}: será feita em duas partes: \uline{por módulos} e \uline{integrada}. Contém
a implementação do sistema: 
\begin{itemize}
\item desenhos do controlador, sistemas de medição e de potência prontos para
testes;
\item desenhos das peças mecânicas a produzir e/ou adquirir
\item Lista de materiais
\begin{itemize}
\item Na fase final do projeto, deverá incluir os PCBs
\end{itemize}
\item Iteração 1: 2 semanas
\item Iteração 2: 1 semana
\end{itemize}
\item \href{Testes}{./Testes/}: será feito em duas partes: \uline{por módulos} e \uline{total}. contém
todos os testes realizados ao sistema e aos vários
protótipos. Consideram-se testes apenas os que são realizados sobre
qualquer componente ou protótipo físico.
\begin{itemize}
\item Iteração 1: 2 semanas
\item Iteração 2: 1 semana
\end{itemize}
\item \textbf{Verificação/Validação}: (3 dias) --- as especificações listadas na
Análise devem ser verificadas e o protótipo validado por um agente externo
(utilizador externo ao grupo).
\item \textbf{Entrega}: (2 semanas) --- término do projeto com:
\begin{enumerate}
\item Protótipo final produzido, verificado e validado
\item Documentação de suporte: como replicar, manual de instruções
\item Relatório Final: \textit{<2020-01-30 Thu>}
\item Apresentação Pública: \textit{<2020-01-31 Fri>}
\end{enumerate}
\end{enumerate}
\item Definição das equipas de projeto pelas áreas identificadas (a \textbf{negrito}
encontram-se os coordenadores de cada departamento):
\begin{itemize}
\item \uline{Mecânica}: \textbf{José Pires}
\item \uline{Human Machine Interface (HMI) \& Sensores}: \textbf{Hugo Carvalho}, Hugo Ferreira
\item \uline{Actuação}: \textbf{João Faria}, Nuno Rodrigues
\item \uline{Controlo}: \textbf{João Carvalho}, Nuno Rodrigues
\begin{itemize}
\item Estas equipas poderão sofrer alterações conforme o projecto assim o
exija.
\end{itemize}
\end{itemize}
\end{enumerate}
\end{enumerate}

\section{\href{/Users/zemiguel/Documents/Univ/MI\_Electro/Sem6/LPI2/PI/github/Deliverables/Final/sec/cheatsheet.pdf::3}{Cheatsheet}}
\label{sec:orgead43b2}
\section{{\bfseries\sffamily ▭▭ IN-PROGRESS} Write Document [8/11]}
\label{sec:org138e41c}
\begin{enumerate}
\item{$\boxtimes$} Introduction
\item{$\boxtimes$} State of the Art
\item{$\boxtimes$} Theorethical Foundations [2/2]
\begin{enumerate}
\item{$\boxtimes$} Project methodologies
\item{$\boxtimes$} Communications
\begin{enumerate}
\item{$\boxtimes$} Bluetooth
\item{$\boxtimes$} Wi-Fi
\item{$\boxtimes$} GPRS
\item{$\boxtimes$} Network Programmings - Sockets
\item{$\boxtimes$} Client-Server Architecture
\end{enumerate}
\end{enumerate}
\item{$\boxtimes$} Requirements Elicitation and Specifications Definition
\item{$\boxtimes$} Analysis
\item{$\boxtimes$} Design
\item{$\square$} Implementation
\item{$\square$} Testing
\item{$\boxtimes$} Verification and Validation
\item{$\square$} Conclusions
\item{$\boxtimes$} Appendices [1/1]
\begin{itemize}
\item[{$\boxtimes$}] Planning
\end{itemize}
\end{enumerate}

\section{12 July}
\label{sec:orgf96db10}
\subsection{Implem [0/1]}
\label{sec:org4f8bf1b}
\begin{itemize}
\item[{$\square$}] Controlo [0/3]
\begin{itemize}
\item[{$\square$}] \uline{Nuno}: Incluir valores dos parametros e do T numa tabela
\item[{$\square$}] \uline{Alexandre}: Algoritmo de velocidade modificado (ver eq. 5.14)
\begin{itemize}
\item Threads:
\begin{enumerate}
\item Simulator (Car Model): generates stimulus - \uline{Alexandre}
\item Controller (PID controllers): receives stimulus and calculates
response: \uline{Nuno}
\begin{enumerate}
\item Left wheel
\item Right Wheel
\end{enumerate}
\end{enumerate}
\end{itemize}
\item[{$\square$}] \uline{Ferreira}: Fazer diagrama de Blocos Simulador + Controlador
\begin{itemize}
\item Documentar, dizendo que o simulador representa o modelo, gerando
estimulos para o controlador, a fim de se analisar o seu comportamento.
\end{itemize}
\end{itemize}
\end{itemize}
\subsection{Testing [0/2]}
\label{sec:orgd1c10f9}
\begin{itemize}
\item[{$\square$}] Controlo: à medida que se faz a implementacao, fazer os testes para
validar o algoritmo de controlo, usando as referencias da simulacao (seccao
5.1.X)
\item[{$\square$}] Bluetooth: testar a aplicacao e documentar
\end{itemize}
\subsection{Validation [0/1]}
\label{sec:org1ad11ef}
\begin{itemize}
\item[{$\square$}] Faria: ask someone to try Smartphone App, take pics and document
\end{itemize}

\section{Functionalities [0/3]}
\label{sec:org6ca3cb2}
\begin{itemize}
\item[{$\boxminus$}] RVVS
\begin{itemize}
\item[{$\boxtimes$}] Wi-Fi: client and server
\item[{$\boxminus$}] Image Aquisition
\begin{itemize}
\item[{$\boxtimes$}] Image capture from webcam
\item[{$\square$}] Video capture and streaming
\end{itemize}
\item[{$\boxtimes$}] Telemetry
\end{itemize}
\item[{$\square$}] NVS
\item[{$\square$}] Smartphone
\end{itemize}
\end{document}